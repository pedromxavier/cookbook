\preface%%
    A intenção deste livro é reunir exercícios interessantes para um curso de introdução à computação na forma de pequenos projetos. Conta com enunciados elaborados e desenvovimentos longos, a fim de explorar alguns dos conceitos básicos de programação. \par

    Este texto é fruto das minhas atividades de monitoria nos cursos de Computação I (MAB125) na Universidade Federal do Rio de Janeiro (UFRJ) entre os anos de 2018 e 2019. Entre as turmas que trabalhei estão os cursos de Engenharia Naval, Engenharia de Produção e o Bacharelado em Ciências Matemáticas e da Terra (BCMT). Por conta disso, você deve encontrar referências a temas bastante diversos, dadas as inúmeras aplicações distintas que foram exercitadas por mim e pelos professores nestes diferentes contextos. \par

    O curso é pensado para alunos que acabaram de ingressar na Graduação e que possivelmente não tiveram contato com computação anteriormente. \par

    Vamos utilizar como referência a linguagem Python. No entanto, este material não pretende ser um curso ou referência da linguagem. Ela nos dará suporte para enunciar os fundamentos de programação, assim como noções de algoritmos e estruturas de dados. De maneira geral, as especificidades da linguagem são discutidas com maior profundidade nos tópicos finais de cada capítulo. \par

    A tipografia foi elaborada no sistema \LaTeX e fortemente inspirada no \textit{layout} da terceira edição do livro \textit{Linear Algebra and its Applications}\cite{strang:1988}, de Gilbert Strang, um dos meus livros favoritos na jornada acadêmica. \par

    \prefacenote{Petrópolis, março de 2021}