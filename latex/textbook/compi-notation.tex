\chapter*{Notação}%%
	Como referência, a notação matemática segue o padrão encontrado na maioria dos livros e artigos do assunto. No entanto, principalmente na hora de falar de código utilizaremos alguns recursos tipográficos para auxiliar na comunicação. \par

	\section*{\textit{Python}}%%
	Quando falando sobre os elementos da linguagem, estes aparecerão em fonte monoespaçada, tal como \code{x}, \code{y} e \code{math}.
	As palavras reservadas (comandos) são destacadas como em \stmt{if}, \stmt{return} e \stmt{del}. Funções, tipos e exceções da biblioteca padrão são mostrados como \type{list}, \type{range}, \type{open}, etc. Por fim, temos as \textit{strings}, \str[']{Oi Mundão}, \str{Adeus, Mundinho}; e os comentários, \lstpy{# Voltei Mundão}. Trechos de código serão formatados como a seguir:
	%%
	\begin{lstpython}
	import math

	def f(n):
		if n == 0 or n == 1:
			return 1
		else:
			return f(n - 1) + f(n - 2)

	print("f(10) =", f(10))
	\end{lstpython}
	%%
	Sessões do console interativo aparecem de maneira bastante similar:
	%%
	\begin{lstpython}
	>>> L = [2, 3, 5, 7, 11]
	>>> L[1]
	3
	>>> L.append(13)
	>>> L
	[2, 3, 5, 7, 11, 13]
	\end{lstpython}
	%%
