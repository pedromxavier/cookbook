\chapter*{Notação}%%
	Como referência, a notação matemática segue o padrão encontrado na maioria dos livros e artigos do assunto. No entanto, principalmente na hora de falar de código utilizaremos alguns recursos tipográficos para auxiliar na comunicação. \par

	\section*{Python}%%
	Quando falando sobre os elementos da linguagem, estes aparecerão em fonte monoespaçada, tal como |x|, |y| e |math|.
	As palavras reservadas (comandos) são destacadas como em |if|, |return| e |del|. Funções, tipos e exceções da biblioteca padrão são mostrados como |list|, |range|, |open|, etc. Por fim, temos as \textit{strings}, |'Oi Mundão'|, |"Adeus, Mundinho"|; e os comentários, |# Voltei Mundão|. Trechos de código serão formatados como a seguir:
	%%
	\begin{lstpython}
	import math

	def f(n):
		if n == 0 or n == 1:
			return 1
		else:
			return f(n - 1) + f(n - 2)

	print("f(10) =", f(10))
	\end{lstpython}
	%%
	Sessões do console interativo aparecem de maneira bastante similar:
	%%
	\begin{pyprompt}
	>>> L = [2, 3, 5, 7, 11]
	>>> L[1]
	§3§
	>>> L.append(13)
	>>> L
	§[2, 3, 5, 7, 11, 13]§
	\end{pyprompt}
	%%
