\chapter[c:numeros]{Números}

    Este capítulo é opcional.

    \section*{Inteiros}

    Além de fazer contas

    \section*{Ponto-flutuante}
    
    
	
    Agora é importante falar um pouco mais sobre o nosso antigo conhecido, o |float|. Existe um padrão industrial, o \textbf{IEE 754}, que especifica como deve ser feita a representação numérica em ponto-flutuante. \par
    
    O padrão afirma que o |float| deve possuir três partes:	o \textit{bit} do sinal, o expoente, e a mantissa (ou fração). O número é positivo quando o primeiro \textit{bit} é 0, e negativo quando é 1. O tamanho do expoente e da mantissa depende de diversos fatores, como o projeto do processador, a linguagem de programação e até mesmo das decisões do programador. \par
    
    \begin{figure}[h]
        \centering
        \small
        |0 01111000 0100000000000000000000|
        \caption{O número 0.15625 no padrão IEE 754, precisão simples.}
        \label{f:iee754}
    \end{figure}
    
    É possível fazer uma analogia com a notação científica. Vamos observar o número de moléculas em um mol de uma substância qualquer:
        $$1 \text{mol} = 6.02 \times 10^{23}$$
    De maneira imediata, podemos separar esse número em três partes também: o sinal ($+$), o expoente (23), e a mantissa (6.02). \par
    
    É importante entender que as representações em ponto-flutuante tem suas limitações. Algumas delas são:
    
    \begin{itemize}
        \item Somar ou subtrair números muito grandes com números muitos pequenos resultará numa \textit{adição ou subtração catastrófica}, isto é, o número menor pode acabar sendo ignorado durante a operação.
    
        \item A mantissa é composta pela soma de potências de $\frac{1}{2}$, portanto números que não são resultado de somas finitas dessas potências vão apresentar erros de representação.
    \end{itemize}

    \section*{Complexos}

    \section*{Vetores}
    
    