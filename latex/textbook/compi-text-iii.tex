\chapter[c:strings]{\textit{Strings} e Texto}
    
    De modo geral, \textit{strings} são o principal tipo básico presente nas linguagens de programação para representar textos. Os literais que representam texto em Python são aqueles compreendidos entre aspas simples |'| ou duplas |"|. Não existe preferência absoluta por uma ou outra, fica a seu critério qual utilizar. No entanto, uma opção pode ser feita pela praticidade ao trabalhar com texto que contenha algum destes caracteres.\par

    \begin{pyprompt}
        >>> historia = "Era uma vez..."
        >>> print(historia)
        §Era uma vez...§
    \end{pyprompt}

    \section*{Codificação}%%
    Vamos discorrer rapidamente sobre os aspectos fundamentais da codificação de texto em computadores. As cadeias de caracteres ou simplesmente \textit{strings} nada mais são do que sequências que comportam um tipo específico de dado, um número a representar determinado símbolo.\par

    Existem diversos "alfabetos" para a codificação de caracteres, sendo a \emph{Tabela ASCII}\footnote{\textit{American Standard Code for Information Interchange.}} desenvolvida na década de 1960 uma das precursoras dentre estas tecnologias. De fato, os padrões de codificação amplamente utilizados na atualidade configuram extensões do seu conjunto inicial de 128 símbolos.\par

    Como um byte possui 8 bits e somente 7 são necessários para representar todas as entradas da tabela, sobra um bit para usos outros, a depender da implementação em voga. O padrão Unicode\cite{unicode:2020}, particularmente em sua especificação \emph{UTF-8}\footnote{\textit{Unicode Transformation Format} em 8 bits.}, utiliza este bit adicional para indicar se serão necessários bytes adicionais na codificação ou não. Precisamente, o último bit indica se determinado caractere está presente nos 128 itens da \emph{Tabela ASCII}.\par

    O \emph{UTF-8} se encontra em cerca de 97\% das páginas da internet\cite{w3techs:2021}. O padrão utiliza de um a quatro bytes, o que permite representar mais de 2000 símbolos com apenas 2 bytes, o suficiente para cobrir todos os alfabetos latinos, do grego ao cirílico. Com os bytes extras codifica-se o chinês, tradicional e moderno, o coreano, o japonês, o sânscrito e a maioria das línguas orientais. Com quatro bytes é possível descrever símbolos cuneiformes, hieróglifos e sua encarnação contemporânea: os emoji.\par

    \begin{problem}{Transformações ASCII}%%
    Da maneira como foi construída, a Tabela ASCII nos permite realizar transformações comuns ao texto de uma maneira bem simples. No Python, é proposta uma interface com a tabela através das funções |ord| e |chr|. |ord| recebe um símbolo e diz qual é sua posição na tabela de codificação através de um número inteiro. Já |chr| nos diz qual caractere ocupa a posição dada pelo inteiro que recebe. Ou seja, uma é inversa da outra, de forma que |chr(ord(c)) = c| para qualquer símbolo |c|. Naturalmente, também vale a recíproca.\par

    \proposal Procure entender a disposição das letras de \emph{a} a \emph{z} e utilizando as funções |ord| e |chr| defina um par de funções que transforme textos quaisquer em sequências de caixa alta e caixa baixa, respectivamente.

    \end{problem}

    \section*{Operações em texto}

    \section*{Formatação e Interpolação}

    \section*{Expressões Regulares}