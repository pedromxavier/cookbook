\chapter{Tipos, Variáveis, Operadores e Funções}%%

	\section*{Introdução}%%
	É interessante, de fato, iniciar o estudo da computação através das funções. As funções trazem consigo os conceitos de tipos e variáveis de maneira muito natural. De maneira simples, uma função pode ser compreendida como a transformação dos dados de entrada nos dados de saída, cada qual com seus respectivos tipos. \par
	
	A noção de tipo está também relacionada à ideia de conjunto. Ser de um determinado tipo é pertencer a um conjunto, mais especificamente ao conjunto daquele tipo. Isto fica bem claro quando se pensa nos números, embora seja um conceito adequado a uma miríade de situações. \par
	
	Em textos de matemática a nível superior, é comum que as funções recebam outros nomes a depender do contexto. Uma nomenclatura um tanto elucidativa é chamar funções de \textit{aplicações}, em inglês, \textit{mappings}. É de bom gosto observar que as funções são \textit{mapas} que levam de um ponto em um conjunto noutro ponto de outro conjunto. \par

	Dito isto, é importante pensar por um tempo no significado de uma função, ter em mente que qualquer processo computacional, por mais complicado que seja, pode ser representado pela aplicação de diferentes funções em sequência. Formalmente, uma transformação será descrita pela composição de diversas transformações mais simples. Isso tudo, porém, no mundo das ideias. \par

	Os \textit{operadores}, por outro lado, são a materialização da computação que desejamos realizar. Operações tratam da maneira explícita como uma tarefa lógica ou aritmética deve ser encadeada. Pode-se dizer, portanto, que as operações estão entre as funções mais simples e mais próximas da realidade. O que acabo de dizer aparece de maneira clara no projeto dos processadores digitais. A arquitetura x86 moderna possui algumas centenas de instruções, dentre operações aritméticas, de acesso à memória e de controle. Um processador RISC, por sua vez, chega a operar com apenas 34 instruções\cite{arm:1995}. O importante é ter em mente que as mais complexas atividades realizadas por um computador são compostas a partir de um conjunto reduzido de operações básicas. \par

	\section*{Tipos}%%
	A teoria por trás dos tipos é bastante extensa e é tratada com detalhe muito maior no volume de Computação II\cite{xavier:2021}. No decorrer dos próximos capítulos vamos tratar das especificidades e casos de uso dos tipos básicos. Por enquanto, vamos nos ater à sua apresentação e ao uso prático inicial. \par

	Vamos começar pelos tipos numéricos. Estão disponíveis os tipos |int|, |float| e |complex|, construídos para representar os números inteiros ($\Z$), reais ($\R$) e complexos ($\C$), respectivamente. A primeira vista, podemos distingui-los pela sintaxe: os números representados em ponto-flutuante%%
	%%
	\footnote{Uma discussão mais aprofundada sobre ponto-flutuante e o padrão IEE 754 se encontra no capítulo \ref{c:numeros}.}%%
	%%
	podem ser escritos adicionando a parte decimal após o ponto, como em |0.5| e |-1.2| ou através de notação científica, como |6.02e+23| ($6,02 \times 10^{23}$). Os números complexos são simplesmente uma extensão do tipo |float|, onde a parte imaginária é indicada acrescentando um |j| ao final do número. A unidade imaginária $\ii$, por exemplo, escreve-se |1j|. Tendo em mãos estes três tipos, podemos realizar as operações aritméticas básicas de adição |+|, subtração |-|, multiplicação |*| e divisão |/|; além da exponenciação |**| e da divisão inteira, realizada através dos operadores de quociente |//| e resto |%|. \par

	Tome um tempo para assimilar o comportamento destes operadores, procurando explorar tanto os casos simples quanto os mais problemáticos como a divisão por zero ou operações entre números grandes demais. Atente, em cada operação para os tipos dos termos assim como para o tipo do resultado. A própria sintaxe deve mostrar-se suficiente para identificar o tipo mas você pode sempre contar com a função |type|.\par
	%%
	\begin{pyprompt}
	>>> 1 + 1
	§2§
	>>> 1 + 1.0
	§2.0§
	>>> type(1 + 1)
	§<class 'int'>§
	>>> type(1 + 1.0)
	§<class 'float'>§
	\end{pyprompt}
	%%
	Se tem uma coisa fundamental a ser percebida neste primeiro exemplo é a hierarquia existente entre os tipos |int|, |float| e |complex|, relação que vamos relembrar através da adaptação de um velho diagrama (figura \ref{f:conjuntos}).
	%
	\begin{figure}[h]
		\centering
		\medskip
		\inputikz[0.5\textwidth]{conjuntos}
		\caption{Conjuntos numéricos}
		\label{f:conjuntos}
	\end{figure}
	%
	Pensando na correspondência entre os tipos e os conjuntos que representam, podemos afirmar que em uma operação entre números de tipos distintos o resultado será do tipo mais geral, ou seja, do conjunto mais externo. É possível também converter números de um tipo em outro, salvo algumas observações. Números inteiros podem vir a ser representados em ponto-flutuante ou como complexo. Isso ocorre da mesma maneira que um |float|, onde o número original passa a ser a parte real de um novo número, com parte imaginária igual a zero. O caminho contrário apresenta a mesma naturalidade. A conversão de um número decimal para inteiro naturalmente descarta a parte fracionária, de onde restam apenas as unidades. Transformar um número complexo em um tipo simplesmente real seria ambíguo, uma vez que não é óbvio que atitude tomar em relação à parte imaginária do número. Portanto, não é possível converter um número complexo para nenhum dos outros tipos. Resumimos este diálogo na figura \ref{f:mapa-numeros}.
	%%
	\begin{figure}[h]
		\centering
		\medskip
		\inputikz[0.5\textwidth]{mapa-numeros}
		\caption{Mapa de tipos numéricos}
		\label{f:mapa-numeros}
	\end{figure}
	%%
	Por fim, vale ressaltar que o tipo |complex| só é utilizado em aplicações bastante específicas, enquanto o |float| está presente na maioria dos casos de uso. O tipo inteiro estará sempre presente, pois desempenha muitas tarefas importantes para além da aritmética. Falaremos de maneira mais precisa acerca dos tipos numéricos e suas aplicações no capítulo \ref{c:numeros}.\par

	Seguindo adiante, vamos tratar de alguns tipos de sequências. Temos disponíveis três tipos básicos de estruturas sequenciais: listas |list|, tuplas |tuple| e \textit{strings} |str|. As duas primeiras são estruturas de uso geral e servem para armazenar de maneira ordenada elementos de qualquer tipo, incluindo os números e até mesmo outras sequências. As \textit{strings} são projetadas para armazenar sequências de caracteres, ou seja, texto. \par

	As listas são delimitadas por colchetes |[]| e seus elementos são separados por vírgulas. As tuplas, por sua vez, utilizam parênteses |()| e tem um comportamento semelhante. As diferenças entre as duas serão esmiuçadas adiante, no capítulo \ref{c:estruturas}. Por fim, as \textit{strings} são compostas pelos caracteres inseridos entre aspas, sejam elas simples |''| ou duplas |""|.\par

	Assim como no caso dos números, é possível transformar listas em tuplas e vice-versa, utilizando as funções |list| e |tuple|. As transformações envolvendo as \textit{strings} e seu tipo |str|, no entanto, possuem um comportamento distinto. Qualquer elemento de qualquer tipo pode ser transformado em \textit{string}, o que consiste em obter uma representação visual para o mesmo. O processo contrário, de transformar \textit{strings} em outros tipos deve ser analisado caso-a-caso. \par

	A operação fundamental entre sequências é a concatenação |+|. Não é possível, contudo, concatenar cadeias de tipos distintos.

	\begin{pyprompt}
	>>> [1, 2, 3, 4, 5]
	§[1, 2, 3, 4, 5]§
	>>> 1, 2, 3, 4, 5
	§(1, 2, 3, 4, 5)§
	\end{pyprompt}
	
	\section*{Variáveis e atribuição}%%
	\begin{lstpython}
	>>> x = 0
	>>> y = 1
	>>> x + y
	1
	\end{lstpython}

	\section*{Funções}%%
	Para definir uma função vamos utilizar o comando |def|, seguido do nome da função, dos seus parâmetros e, por fim, o código que deve executar.
	\begin{lstpython}
	>>> def f(x, y):
	...		return x + y
	...
	>>> z = f(2, 3)
	>>> z
	5
	\end{lstpython}

    \begin{problem}[p:calculo:1]{Cálculo I - Limites}
	O limite $\ell$ de uma função $f(x)$ no ponto $a$ é o valor que a função assume conforme $x$ se aproxima de $a$. Escrevemos:
		$$\ell = \lim_{x \to a} f(x)$$
	Quando a função não apresentar nenhuma descontinuidade ao redor do ponto $a$, podemos afirmar com tranquilidade que $\ell = f(a)$. Uma maneira simples de aproximar os limites laterais no computador é calcular
		$$\lim_{x \to a^{-}}f(x) \approx f(a - \epsilon) \text{ e } \lim_{x \to a^{+}}f(x) \approx f(a + \epsilon)$$
	escolhendo um valor suficientemente pequeno para $\epsilon$. 

	\begin{lstpython}
	def lim(f, a, e=0.01):
		"""Limite da função f no ponto a."""
		e = 1E-4 # 0.0001
		return (f(a - e) + f(a + e)) / 2.0
	\end{lstpython}

	\proposal Faça uma função |lim(f, a)| para calcular o limite da função |f| no ponto |a|. Leve em consideração a existência do limite.
	\end{problem}

	\begin{problem}[p:musica:1]{Música I - As notas e os sons}
		Cada nota musical corresponde a uma frequência distinta (em Hz). Tomando o lá central (A4) como referência, em 440Hz, podemos calcular a frequência das outras notas com base na distância relativa a essa nota.
        $$f(n) = 440 \times 2^{(n/12)}$$
        Na tabela abaixo, vemos as notas musicais, seus símbolos, e a distância em semitons\footnote{Dois semitons equivalem a um tom. No violão, cada casa de uma mesma corda está a um semitom da casa adjacente. No piano, quando há uma tecla preta entre as brancas, há uma distância de um tom entre elas. Quando a tecla preta não está, a distância é de meio tom, ou um semitom.} para o lá central (A4).

		\begin{center}
            \begin{tabular}{|l|c c c c c c c c c c c|}
                \hline
                Símbolo &  & F3 & G3 & A4 & B4 & C4 & D4 & E4 & F4 & G4 & \\
                Nome & ... & fá & sol & lá & si & dó & ré & mi & fá & sol &  ... \\
                Semitons &  & -4 & -2 & 0 & 2 & 3 & 5 & 7 & 8 & 10 & \\
                \hline
            \end{tabular}
        \end{center}

		Você pode notar que a cada 12 semitons, a nota se repete com o dobro da frequência. Chamamos este intervalo entre notas de oitava. Na notação acima, a cada letra indica uma nota diferente, enquanto o número diz a oitava em que ela se encontra. \par
        
		\proposal
        Faça uma função |f(n)| que retorne a frequência em Hertz de uma nota que se encontra a $n$ semitons de distância do lá central. Arredonde o resultado para o número inteiro mais próximo usando a função |round(numero, dígitos)|. \par
        
        \begin{lstpython}
	>>> f(0), f(2), f(3)
	(440, 494, 523)
	>>> f(12) # frequência dobra ...
	880
        \end{lstpython}
	\end{problem}

    \begin{problem}{Coordenadas polares}
        
        Estamos acostumados a pensar em coordenadas cartesianas na hora de descrever a geometria de um determinado objeto. No entanto, o sistema de coordenadas deve ser escolhido conforme o cenário em que se está trabalhando.
        %%
		
        \inputikz{cartesiano}

		%%
        O ponto $(4, 3)$, quando escrito em coordenadas polares, nos dá:
        \begin{align*}
        r &= \sqrt{4^2 + 3^2} = \sqrt{16 + 9} = \sqrt{25} = 5\\
        \theta &= \arctan\frac{3}{4} = 0.6435 \text{ rad} \approx 36.87^{\circ}
        \end{align*}
        
        \proposal
		Construa duas funções: \texttt{polar(x, y)} levará um ponto em coordenadas cartesianas $(x, y)$ para a forma polar $(r, \theta)$ e \texttt{cart(r, theta)}, que fará o caminho contrário.\par
        
        \begin{lstpython}
		>>> import math
		>>> polar(-1, 0)
		(1.0, 3.141592653589793)
		>>> cart(2, math.pi)
		(-2.0, 0.0)
        \end{lstpython}
        
    \end{problem}

	\section*{Condicionais}%%
	Uma das ferramentas mais simples e poderosas dos computadores é a sua capacidade de tomar diferentes atitudes mediante uma condição.

	\section*{Repetição}%%

	\begin{problem}{Sequência de \emph{Collatz}}    
	A sequência de \emph{Collatz} é obtida aplicando sucessivamente a função
	{\large
		\begin{align*}
		f(n) = \begin{cases}
		3\, n + 1, &\text{ se } n \text{ for ímpar}\\
		n \div 2, &\text{ se } n \text{ for par}
		\end{cases}
		\end{align*}
	}
	Por exemplo, começamos com $n = 26$. Após sucessivas aplicações temos:
	$$26 \to 13 \to 40 \to 20 \to 10 \to 5 \to 16 \to 8 \to 4 \to 2 \to 1$$
	Isso nos dá uma sequência com $11$ números. $40$ é maior que $26$, mas sua sequência só teria $9$ números.\par

	Ainda não se sabe se todos os números induzem uma sequência que termina em $1$. No entanto, até agora não foi encontrado um número sequer em que isso não tenha acontecido! \par

	\proposal Faça uma função que calcule o comprimento da sequência gerada a partir de um número natural $n$ qualquer. \par
	
	\begin{lstpython}
	>>> collatz(26)
	11
	>>> collatz(40)
	9
	>>> collatz(1)
	1
	\end{lstpython}
	\end{problem}

	\begin{problem}{\textit{Chatbot} I}
	Existe uma história corrente de que o grande motor da inteligência artificial em escala industrial é o uso indiscriminado do comando |if| e seus derivados. Não é difícil pensar que através de um número suficientemente grande de condições seja possível abarcar uma grande variedade de comportamentos. \par

	\proposal Construa uma função chamada |chatbot|
	\end{problem}

	\section*{Recursividade}%%
	
	\section*{Parâmetros}%%
	
	\section*{Decoradores}%%
	
	\endinput