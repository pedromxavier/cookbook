%% Usando cores pasteis em Português do Brasil
\documentclass[br, pastel, round]{week}

%% Título e subtítulo
\title{Horários do Pedro 2020.2 {\Large v1.2}}
\subtitle{(22 mar. - 12 jun. 2021)}

%% Horas do seu dia
\sethours{5}{22}

\begin{document}
    \begin{week}
        %% A hora é simplesmente o número da hora que a atividade acontece.
        %% Os dias são informados por números:
        %%  0 - Domingo
        %%  1 - Segunda
        %%  2 - Terça
        %% e assim por diante.

        %% \xtask[<cor>]{<titulo>}{<dia>}{<dias>}{<hora>}{<horas>}
        \xtask[brown]{Café \& Planejamento}{1}{5}{6}{2}%
        \task[brown]{Planejamento}{0}{19}{2}%

        %% \xtask[<cor>]{<titulo>}{<dia>}{<hora>}{<horas>}
        \task[cyan]{Redes II}{1}{10}{2}%
        \task[cyan]{Redes II}{3}{10}{2}%

        \task[darkpurple]{Computação FAU}{3}{12}{1}%

        \task[darkpurple]{MONITORIA\\~\\ Computação FAU\\ \&\\ Computação Gráfica}{1}{14}{3}

        %% Em nomes grandes pode ser preciso usar '\\' para quebra de linha.
        \task[blue]{Qualidade de\\ Software}{2}{10}{2}%
        \task[blue]{Qualidade de\\ Software}{4}{10}{2}%
        

        %% Use esse comando para atividades que se expandem horizontalmente.
        \xtask[brown]{Almoço}{1}{2}{12}{1}%
        \xtask[brown]{Almoço}{3}{1}{13}{1}%
        \xtask[brown]{Almoço}{4}{2}{12}{1}%

        \task[yellow]{Álgebra Linear III}{2}{13}{2}%
        \task[yellow]{Álgebra Linear III}{4}{13}{2}%
        
        %% Também é possível usar horas quebradas sem problemas.
        \task[darkgreen]{Empreendedorismo}{5}{13.5}{4}%

        \task[deepblue]{Descanso}{5}{18}{5}%
        \task[deepblue]{Descanso}{6}{5}{13}%

        \task[green!50!gray]{Arrumação}{6}{19}{2}%
        \task[green!50!gray]{Arrumação}{5}{10}{2}%
        \task[green!50!gray]{Arrumação}{4}{15}{2}%

        \task[orange]{Reunião\\ Priscila}{2}{15}{2}%

        \task[red]{Programação\\ Avançada}{1}{17}{2}%
        \task[red]{Programação\\ Avançada}{3}{17}{2}%
        
        \task[orange]{PowerPointers}{1}{20}{1.5}
    \end{week}
\end{document}